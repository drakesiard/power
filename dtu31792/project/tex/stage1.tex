\documentclass[11pt,a4paper]{article}
\usepackage[numbers]{natbib}
\usepackage{graphicx}
\usepackage{subcaption}
\usepackage{amsmath}
\usepackage{amsfonts}
\usepackage{mathtools}
\usepackage{enumitem}
\usepackage{setspace}
\usepackage{adjustbox}
\usepackage{placeins}
\usepackage{booktabs}
\usepackage{tabulary}
\usepackage{hyperref}
\usepackage[capitalise,noabbrev]{cleveref}
\usepackage[a4paper, total={6in, 9in}]{geometry}

\newcommand{\pkg}[1]{{\fontseries{b}\selectfont #1}} 

\bibliographystyle{IEEEtranN}

\title{\textbf{Project step 1: deterministic optimisation}}
\author{S. Drake Siard\\
DTU 31792, Spring 2021}
\date{10 Jan 2021}
\begin{document}

\maketitle

\section{Problem}

The convex optimization problem selected in step 0 was the nodal energy market-clearing problem with transport constraints, approximating network power flow by DC linearisation, outlined in \cite{kazempourLectureMarketClearing2021}.
In order to expand the problem to a reasonable size, day-ahead bid and offer data for October 7, 2020 were downloaded from the Midcontinent ISO (MISO) Market Data site \cite{MISOMarketData}.
The data format was interpreted according to the MISO Energy Markets Business Practices Manual \cite{MISOEnergyOperating2020}. However, some simplifications were made:
\begin{itemize}
\item The day-ahead power market was simplified to economic participants and active power only, removing ancilliary services.
\item Non-monotonic bid or offer curves were removed from the data set.
\item Binary variables were avoided, as directed on \cite[p. 14]{kazempourLectureMarketClearing2021}, and therefore: 
\begin{itemize}
\item All block bids were treated as continuous (slope bids)
\item Unit commitment constraints and costs were ignored
\end{itemize}
\end{itemize}


The problem was coded in Python using the MOSEK Fusion API \cite{mosekapsMOSEKFusionAPI2021}.



\renewcommand{\refname}{\section{References}}.
\bibliography{DTU31792}


\end{document}